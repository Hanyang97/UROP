\chapter{Model}

\section{Clustering}
We aim to cluster sessions and predict commands based on the first \(d\) commands of each session.
Therefore, the model can be divided into clustering part and prediction part.
\\\\
We need to assign each session a label, so that we can evaluate how homogeneous each cluster is,
and we can use the labels as surrogates for predictions.
We divide sessions into sequences of commands of length \(l_c\)\cite{sadique2021analysis}.
For example, if a session is \textbf{A-B-C-D-E-F} and \(l_c=3\), 
we will have sequences \textbf{A-B-C}, \textbf{B-C-D}, \textbf{C-D-E}, and \textbf{D-E-F}.
Sequences of commands are ranked by their number of appearances, 
i.e. the most frequent sequence is ranked 1, the second most frequent sequence is ranked 2.
For each session \(x\), we assign label \(y\) as the highest rank which can be achieved by the sequences obtained from \(x\).
We denote \(x_{j}\) to be the \(j\)th command of \(x\),
and \(x_{:d}\) to be the initial \(d\) commands of \(x\).
Suppose we have \(N\) unique initial \(d\) commands and the lowest rank of our sessions is \(K\),
an \(N\) by \(K\) matrix of counts \(A\) can be constructed as following: 
\(A_{pq}\) is the number of sessions \(x\) for which \(x_{:d}\) is the \(p\)th unique first \(d\) commands,
and \(y=q\).
\\\\
Hierarchical Agglomerative Clustering method is used for our clustering.
The idea is that we start with \(N\) clusters, and pairwisely
merge closest clusters in a predefined metric until certain threshold.
In our case, for each initial cluster \(p\), we assume that the unknown categorical probability
$\theta=(\theta_1,\ldots,\theta_K)$ follows a Dirichlet prior,
\begin{equation}
    \theta\sim \textnormal{Dirichlet}(\beta_1,\ldots,\beta_K),
\end{equation}
with each $\beta_j>0$. Hence we can calculate \(P(A_{p,:}|\theta)\).
Let us define \(E\) as the dissimilarity matrix among clusters.
We then use \(E_{pq}=\log(P(A_{p,:}, A_{q,:}|\theta)) - \log(P(A_{p,:}|\theta)) - \log(P(A_{q,:}|\theta))\)
to calculate the gain in similarity after merging the cluster \(p\) and \(q\). 
Finally, we can perform the agglomerative clustering as usual. 
We use minimum linkage for agglomerative clustering.

\section{Prediction}
We now come to prediction after finishing clustering.
Suppose a particular cluster contains $M\leq N$ data points, corresponding to data matrix row positions $i_1,\ldots,i_M$ of the $N\times K$ data matrix of counts $A=(a_{ij})$. For $1\leq j \leq K$, let
\begin{equation}
  n_j = \sum_{\ell=1}^M a_{i_\ell j}
\end{equation}
be the total frequency of category $j$ in the cluster, and let $n_{\mydot} =\sum_j n_j$. Suppose the unknown categorical probabilities $\theta=(\theta_1,\ldots,\theta_K)$ for the cluster follow a Dirichlet prior,
\begin{equation}
  \theta\sim \textnormal{Dirichlet}(\alpha_1,\ldots,\alpha_K),
\end{equation}
with each $\alpha_j>0$ and $\alpha_{\mydot} = \sum_j \alpha_j$.
Then the marginal likelihood of the sequence of observed categories which gave rise to the bin frequencies $n_1,\ldots,n_k$ is
\begin{equation}
  \frac{\Gamma(\alpha_{\mydot})}{\Gamma(\alpha^\ast_{\mydot})}\prod_{j=1}^K \frac{\Gamma(\alpha^\ast_j)}{\Gamma(\alpha_j)},
\end{equation}
where $\alpha^\ast_j=\alpha_j + n_j$ for $1\leq j \leq K$ and $\alpha^\ast_{\mydot}=\sum_j \alpha^\ast_j=\alpha_{\mydot}+n_{\mydot}$.
Furthermore, the posterior distribution for $\theta$ after observing $n_1,\ldots,n_K$ is
\begin{equation}
  \theta\mid n_1,\ldots,n_K \sim \textnormal{Dirichlet}(\alpha^\ast_1,\ldots,\alpha^\ast_K).
\end{equation}
In particular, this means
\begin{equation}
  \mathbb{E} (\theta_j \mid n_1,\ldots,n_K) = \frac{\alpha^\ast_j}{\alpha^\ast_{\mydot}}.\label{eq:predictive}
\end{equation}
The quantity \eqref{eq:predictive} is also the predictive probability for category $j$ in that cluster. So if we have a new observation assigned to the cluster with label $y_{N+1}$, the prediction score we get from that observation is
\begin{equation}
  \frac{\alpha^\ast_{y_{N+1}}}{\alpha^\ast_{\mydot}}.
\end{equation}
For each leaf node, the expected predictive probability in that cluster is
\begin{equation}
    \mathbb{E} (p(x)) = \sum_{j=1}^K (\frac{\alpha^\ast_j}{\alpha^\ast_{\mydot}})^2.
\end{equation}
Therefore, we can calculate the weighted sum of the expected predictive probabilities across the clusters,
where the cluster weight is proportional to the number of observations in that cluster.
This can be an indication of our overall cluster result.

\section{Model Architecture}
We can now combine the clustering and prediction part to construct our model.
Suppose we want to train depth \(d\), 
which means we know the initial \(d\) commands of all sessions to help us cluster.
\begin{figure}[h]
    \centering
    \scalebox{.8}{\tikzset{%
  % Specifications for style of nodes:
            base/.style = {rectangle, rounded corners, draw=black,
                           minimum width=4cm, minimum height=1cm,
                           text centered, font=\sffamily},
  activityStarts/.style = {base, fill=blue!30},
       startstop/.style = {base, fill=red!30},
    aggclustering/.style = {base, fill=green!30},
         process/.style = {base, minimum width=2.5cm, fill=orange!15,
                           font=\ttfamily},
}
% Drawing part, node distance is 1.5 cm and every node
% is prefilled with white background

\begin{tikzpicture}[node distance=1.5cm,
    every node/.style={fill=white, font=\sffamily}, align=center]
  % Specification of nodes (position, etc.)
  \node (start)             [activityStarts]              {Train depth \(d\)};

  \node (ncluster)      [process, below of=start, yshift=-0.25cm] 
  {Sessions with same initial \(d\) commands form a cluster.\\Total \(N_d'\) clusters};
  \node (matA)     [process, below of=ncluster, yshift=-0.25cm]          
  {Find matrix of counts $A$ by using \(d\) initial commands};
  \node (matE)      [process, below of=matA, yshift=-0.5cm]   
  {Find similarity matrix $E$};
  \node (aggclustering2)      [aggclustering, yshift=-0.5cm,below of=matE]  
  {Agglomerative Clustering, total \(N_d\) clusters};
  \node (end)      [startstop, yshift=-0.25cm,below of=aggclustering2]  
  {Analyse the expected predictive probability and end training};
  % Specification of lines between nodes specified above
  % with aditional nodes for description 
  \draw[->]      (start) -- (ncluster);
  \draw[->]      (ncluster) -- (matA);
  \draw[->]     (matA) -- node[]{
      Multinomial-Dirichlet with \(\boldsymbol{\beta}\) as Dirichlet prior parameter
  }(matE);
  \draw[->]      (matE) -- node[]{Decision threshold}(aggclustering2);

  \draw[->]      (aggclustering2) -- (end);
\end{tikzpicture}}
    \caption{Flowchart of Intuitive Model}
    \label{fig:model1}
\end{figure} 
The intuitive way of combination is like Fig \ref{fig:model1}. 
However, the main drawback is that it cannot learn from previous clustering.
For example, the initial 7 commands \textbf{A-B-C-D-E-F-G} and \textbf{A-B-C-D-E-F-H}
are completely different in the intuitive model. 
The initial 6 commands, however, are considered same.
Many times the command \textbf{G} and \textbf{H} may only differ by a URL.
Treating \textbf{A-B-C-D-E-F-G} and \textbf{A-B-C-D-E-F-H} and \textbf{I-I-I-I-I-I-I} 
as 3 completely different cluster is not reasonable.
\\\\
Therefore, we propose a model which can both combine clustering and prediction, and learn from previous depth.
\begin{figure}[h]
    \centering
    \scalebox{.8}{\tikzset{%
  % Specifications for style of nodes:
            base/.style = {rectangle, rounded corners, draw=black,
                           minimum width=4cm, minimum height=1cm,
                           text centered, font=\sffamily},
  activityStarts/.style = {base, fill=blue!30},
       startstop/.style = {base, fill=red!30},
    aggclustering/.style = {base, fill=green!30},
         process/.style = {base, minimum width=2.5cm, fill=orange!15,
                           font=\ttfamily},
}
% Drawing part, node distance is 1.5 cm and every node
% is prefilled with white background

\begin{tikzpicture}[node distance=1.5cm,
    every node/.style={fill=white, font=\sffamily}, align=center]
  % Specification of nodes (position, etc.)
  \node (start)             [activityStarts]              {Train depth \(d\)};
  \node (matAs)     [process, below of=start, yshift=-0.5cm]          
  {Within each cluster in depth \(d-1\), find matrices of counts $A$\\
  by using \(d\) initial commands of sessions in each cluster};
  \node (matEs)      [process, below of=matAs, yshift=-0.5cm]   
  {Find similarity matrices $E$ corresponding to \(A\)};
  \node (aggclustering)      [aggclustering, yshift=-0.5cm,below of=matEs]  
  {Agglomerative clustering within each previous cluster};
  \node (ncluster)      [process, below of=aggclustering, yshift=-0.25cm] 
  {Total \(N_d'\) clusters};
  \node (matA)     [process, below of=ncluster, yshift=-0.25cm]          
  {Find matrix of counts $A$ by using \(d\) initial commands};
  \node (matE)      [process, below of=matA, yshift=-0.5cm]   
  {Find similarity matrix $E$};
  \node (aggclustering2)      [aggclustering, yshift=-0.5cm,below of=matE]  
  {Agglomerative Clustering, total \(N_d\) clusters};
  \node (end)      [startstop, yshift=-0.25cm,below of=aggclustering2]  
  {Analyse the expected predictive probability and end training};
  % Specification of lines between nodes specified above
  % with aditional nodes for description 
  \draw[->]             (start) -- node[] 
                                    {if \(d>1\), suppose there are \(N_{d-1}\) clusters in depth \(d-1\)}(matAs);
  \draw[->]     (matAs) -- node[]{
      Multinomial-Dirichlet with \(\boldsymbol{\beta}\) as Dirichlet prior parameter
  }(matEs);
  \draw[->]      (matEs) -- node[]{Decision threshold}(aggclustering);
  \draw[->]      (aggclustering) -- (ncluster);
  \draw[->] (start.east) -- ++(5,0) -- ++(0,-7.75)  --                
  node[xshift=2.8cm, yshift=4cm, text width=3cm]
  {if \(d=1\), sessions with same first command form a cluster}(ncluster.east);
  \draw[->]      (ncluster) -- (matA);

  \draw[->]     (matA) -- node[]{
      Multinomial-Dirichlet with \(\boldsymbol{\alpha}\) as Dirichlet prior parameter
  }(matE);
  \draw[->]      (matE) -- node[]{Decision threshold}(aggclustering2);

  \draw[->]      (aggclustering2) -- (end);
\end{tikzpicture}
}
    \caption{Flowchart of Our Model}
    \label{fig:model2}
\end{figure}
Suppose we want to cluster in depth \(d\).
As shown in Fig \ref{fig:model2}, the idea we introduce is to cluster within each previous cluster in depth \(d-1\) first.
After this is done, we try to see if the existing clusters can be further combined.
This assymetric clustering procedure gives advantage to sessions with different initial \(d\) commands,
but in the same cluster in depth \(d-1\).    
