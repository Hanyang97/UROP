\chapter{Introduction}
\section{Objectives}
Artificial intelligence is an active research and development area within cyber-security. 
The potential to automatically detect intruders within a computer network by building sophisticated, 
time-varying statistical models of their evolving behaviours, 
allowing outlying anomalous behaviours to be identified, 
has attracted much interest from industry and government. 
Less attention has been paid to modelling the actions and objectives of network intruders,
largely due to a lack of available data containing known malicious behaviour.
This project will investigate a new data resource, 
where in collaboration with Microsoft a so-called “honeypot” has been planted in the college network 
to entice network intruders and record their actions. 
Discoveries from this research theme have the potential 
to provide enterprise situational awareness of the intentions of an intruder detected on a network, 
allowing a targeted response such as preventing access to the most sensitive data held by an organisation.

\section{Related Work}
Some work \cite{sadique2021analysis}studies ways to predict attackers' next commands from the shell commands
using Levenshtein distance without analyzing network traffic.