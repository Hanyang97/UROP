\chapter{Evaluation}
We set our Dirichlet prior \(\boldsymbol{\alpha}\) and \(\boldsymbol{\beta}\) to be 
0.0001 in all categories, given we have investigated that \(K\),
the size of categories in a cluster, is around 1000 for this dataset.
The cluster threshold we use is \(0.3\).
We also experiment different length of sequence of commands, \(l_c\).
\\\\
In order to evaluate our performance, we design 2 baselines using \(l_c=12\) to calculate matrices \(A\) and \(E\). 
\textbf{Baseline 1} is to keep every unique initial \(d\) commands in its own clusters for depth \(d\).
\textbf{Baseline 2} is to put all observations into a single cluster, i.e. not doing any clustering.
\begin{table}[h]
    \centering
    \begin{tabular}{lllllll}
\toprule
{} &   Depth 1 &   Depth 2 &   Depth 3 &   Depth 4 &   Depth 5 &   Depth 6 \\
\midrule
$l\_c=3$    &  0.476399 &  0.524108 &  0.539341 &  0.547319 &  0.551775 &  0.555026 \\
$l\_c=6$    &  0.460124 &   0.50814 &  0.524318 &  0.532581 &  0.536925 &  0.546313 \\
$l\_c=9$    &  0.453345 &  0.501228 &  0.527811 &  0.536087 &  0.541361 &   0.55031 \\
$l\_c=12$   &  0.450155 &   0.49781 &  0.530307 &  0.538657 &  0.544359 &   0.55324 \\
Baseline 1 &  0.454241 &  0.517255 &  0.549589 &  0.568437 &  0.574061 &  0.603752 \\
Baseline 2 &  0.432368 &        NA &        NA &        NA &        NA &        NA \\
\bottomrule
\end{tabular}

    \caption{Expected predictive probabilities}
    \label{tab:prob_df}
\end{table}
\begin{table}[h]
    \centering
    \begin{tabular}{lllllll}
\toprule
{} & Depth 1 & Depth 2 & Depth 3 & Depth 4 & Depth 5 & Depth 6 \\
\midrule
$l\_c=3$    &      11 &      41 &      66 &      76 &      80 &      89 \\
$l\_c=6$    &      14 &      48 &      71 &      81 &      94 &     104 \\
$l\_c=9$    &      14 &      48 &      73 &      82 &      95 &     101 \\
$l\_c=12$   &      14 &      47 &      69 &      81 &      95 &     102 \\
Baseline 1 &    1796 &    7380 &   14237 &   19537 &   19931 &   28116 \\
Baseline 2 &       1 &      NA &      NA &      NA &      NA &      NA \\
\bottomrule
\end{tabular}

    \caption{Number of clusters}
    \label{tab:cluster_df}
\end{table}
\\
Table \ref{tab:prob_df} shows the result of our expected predictive probabilities.
Table \ref{tab:cluster_df} shows the result of number of clusters.
For Baseline 1, as the number of clusters grows drastically fast as depth increases,
most clusters will have very few elements, except for the few dominant ones.
Therefore, we have many small homogeneous clusters in Baseline 1
and we can expect the expected predictive probabilities increases as depth increases.
For Baseline 2, it is meaningless to talk about depth as we put all observations into a single cluster.
its expected predictive probability is lowest among all, as we expect.
However, even if we put all obervations into 1 single cluster,
the expected predictive probability, 0.432368, is still quite high.
This can be due to the imbalance of dataset which can limit our performance gain as depth increases.
\\\\
A sample clustering of initial commands can be found in Appendix \ref{lst:cluster}.
We can see that for each \(l_c\), the expected probabilities increase as depth increases,
and the numbers of clustering are much fewer than Baseline 1.